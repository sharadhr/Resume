%-------------------------------------------------------------------------------
%	SECTION TITLE
%-------------------------------------------------------------------------------
\cvsection{Experience}
\begin{cventries}
	\cventry{Software and Test Engineer (C++11, AWS SDK, GNU Autotools, CMake, POSIX)} % Job title
	{PetaGene Ltd} % Organization
	{Cambridge, United Kingdom} % Location
	{Aug 2023 -- Present} % Date(s)
	{
		\begin{cvitems} % Description(s) of tasks/responsibilities
			\item Added options to compile cunoFS with Clang and link-time optimisation (LTO) to reduce binary size by 20\% and improve performance by 10\%.
			\item Led migration of Linux cunoFS and Autotools build system to CMake and vcpkg while porting to macOS, improving build and CI times by a factor of 10.
			\item Implemented graphical front-end for cunoFS in C\# using Avalonia UI framework to simplify cross-platform bring-up.
		\end{cvitems}
	}

	%---------------------------------------------------------

	\cventry{Undergraduate Teaching Assistant} % Job title
	{National University of Singapore (NUS) School of Computing (SoC)} % Organization
	{Singapore} % Location
	{Aug 2020 -- Nov 2022} % Date(s)
	{
		\begin{cvitems} % Description(s) of tasks/responsibilities
			\item CS1101S Programming Methodology, CS2100 Computer Organisation, CS3241 Computer Graphics, CS4247 Real-time Rendering.
			\item Conducted weekly tutorials and recitations, prepared materials and videos for students, and marked assignments.
			\item Set up auto-grading harness for computer graphics assignments to automate marking by comparing framebuffers and pixel errors.
		\end{cvitems}
	}

	%---------------------------------------------------------

	\cventry{Embedded Software Engineering Intern (Sensors and IoT Division: C/C++, CMake, STM32)} % Job title
	{Government Technology Agency, Singapore (GovTech)} % Organization
	{Singapore} % Location
	{May 2022 -- Aug 2022} % Date(s)
	{
		\begin{cvitems} % Description(s) of tasks/responsibilities
			\item Implemented a C++ wrapper over Linux Serial Peripheral Interface (SPI) syscall interface. Reduced wheel-reinvention, and improved linkage for other projects using C++.
			\item Implemented firmware on an STM32 microcontroller in C++ to emulate a Trusted Platform Module (TPM) over I\textsuperscript{2}C for Raspberry Pi (rPi). Improved security on the rPi, and saved costs on purpose-built TPMs.
		\end{cvitems}
	}

	%---------------------------------------------------------

	%---------------------------------------------------------
\end{cventries}
