%-------------------------------------------------------------------------------
%	SECTION TITLE
%-------------------------------------------------------------------------------
\cvsection{Projects and Open-Source}
\begin{cventries}
	\cventry{C++20 module for Vulkan-Hpp (C++, CMake)} % Description
	{vulkan.cppm \footnotesize{(\href{https://github.com/KhronosGroup/Vulkan-Hpp/pull/1582}{Merge request})}} % Project title
	{} % leave blank
	{} % leave blank
	{
		\begin{cvitems} % Project description
			\item Modified a code generator to output a C++20 module interface file for the Vulkan-Hpp wrapper library.
			\item Tested with Clang on Linux and MSVC on Windows.
		\end{cvitems}
	}

	\cventry{Compiler for \href{https://ilyasergey.net/CS4212/_static/oat-full.pdf}{Oat language} (OCaml, Menhir)} % Description
	{Oat Compiler} % Project title
	{} % leave blank
	{} % leave blank
	{
		\begin{cvitems} % Project description
			\item Front-end outputs a subset of LLVM IR;\@ back-end compiles IR to a subset of x86\_64 assembly.
			\item Includes compile-time type-checking and optimisations e.g.\ constant folding, dead-code elimination, and register allocation with graph colouring.
		\end{cvitems}
	}

	%---------------------------------------------------------

	\cventry{Quad-core cache-coherence simulator (C++20, CMake)} % Description
	{cache-sim \footnotesize(\href{https://github.com/sharadhr/cs4223-cache-sim}{Repository})} % Project title
	{} % leave blank
	{} % leave blank
	{
		\begin{cvitems} % Project description
			\item Implements MESI, MOESI, and Dragon cache-coherence protocols.
			\item Correctly simulates cache-coherence behaviour of a real quad-core CPU, is configurable (cache size, associativity), and outputs statistics in \texttt{.csv} format.
		\end{cvitems}
	}

	%---------------------------------------------------------

	\cventry{Lexer and parser for a C-like toy language (C++17)} % Description
	{Static Program Analyser} % Project title
	{} % leave blank
	{} % leave blank
	{
		\begin{cvitems} % Project description
			\item Lexer implemented with \texttt{std::regex} state machine; parser is recursive-descent.
			\item Inserts information such as variable declarations, function calls, and control flow into a database about a given program written in the toy language.
		\end{cvitems}
	}

	%---------------------------------------------------------
\end{cventries}
